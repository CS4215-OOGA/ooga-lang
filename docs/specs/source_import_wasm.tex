\subsection*{Import directives}

Import directives allow programs to import values from modules and bind them to names, whose scope
is the entire program in which the import directive occurs.  Import directives can only appear at the top-level.  All names that appear in import directives
must be distinct, and must also be distinct from all top-level variables.

The import graph must not contain cycles.

\subsubsection*{Importable module types}

The module being imported must either be a Source \S 1 WebAssembly module (the usual kind of module) or a Source Imports module.

\subsubsection*{Import filenames}

The module name can be an absolute URL (e.g. \nolinkurl{https://www.example.com/my_modules/module.source}) or a relative URL (e.g. \nolinkurl{std/misc.source}).  If the file extension is ".source", it may be omitted.

An absolute URL will fetch the module from the specified URL.

A relative URL will fetch the module from an implementation-defined location.